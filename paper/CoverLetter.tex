% !TEX TS-program = pdflatexmk

\documentclass{letter}

\usepackage{hyperref}

\signature{Will M. Farr}

\address{Birmingham Institute for Gravitational Wave Astronomy\\School
  of Physics and Astronomy\\University of Birmingham\\Birmingham\\B15
  2TT\\United
  Kingdom\\\href{mailto:w.farr@bham.ac.uk}{w.farr@bham.ac.uk}}

\begin{document}

\begin{letter}{Dr.\ Leslie Sage\\968 National Press Building\\529 14th Street NW\\Washington DC 20045-1938\\United States}

\opening{Dear Dr. Sage:}

Please find enclosed a submission to be considered for a Nature
Letter.  We think these results are particularly exciting because they
are the first time the spin measurements in the existing three likely
gravitational wave events (GW150914, LVT151012, and GW151226) are
analysed together to constrain the formation scenario of the observed
population of merging binary black holes, a long-standing problem in
gravitational wave astronomy.  Additionally, the paper explains the
origin of and corrects a misconception in the existing literature
(particularly the Vitale, et. al (2017) paper referenced in our
submission) that $\mathcal{O}(100)$ detections will be required to
distinguish between different formation scenario in these objects.
[Note (added 19 July 2017): this statement does not accurately reflect
the relationship between Vitale, et al. (2017) and the current paper.
See text.]

Our submission has been formatted using the AASTeX LaTeX package.  We
estimate that the main body of the text, excluding the abstract, is
$\sim 1600$ words long, with four figures, each single-panel, and is
therefore within or very close to the Nature Letter limit.  The
abstract is $\sim 240$ words long (not counting references, which are
``Harvard style'' in the AASTeX package).  The document contains five
sections in an Appendix that we would like to be supplemental
information to the Letter.

\closing{Sincerely,}

\end{letter}

\end{document}
